\documentclass[11pt]{article}
\usepackage[utf8]{inputenc}
\usepackage{amsmath,amssymb,array}
\usepackage{amsthm,alltt}
\usepackage[italian]{babel}
\usepackage{pgf}
\usepackage{hyperref}
%\usepackage[dvips]{graphicx}

\textheight=22cm
\textwidth=16cm
\oddsidemargin=0.25cm
\evensidemargin=0.25cm

\immediate\write18{sh ./vc}
\input{vc}



\title{Esercizi svolti}
\author{Gianluca Della Vedova}
\date{\today, {\tiny revisione \VCRevision}}

\begin{document}

\section{Database universitario}

\subsection{Elenco tabelle}

\noindent
PERSONA (\textbf{ID}, is\_studente, is\_docente, nome, cognome)\\
STUDENTE (\textbf{matricola}, persona references persona(ID))\\
DOCENTE (\textbf{codice}, persona references persona(ID))\\
CORSO (\textbf{ID}, nome)\\
OCCORRENZA\_CORSO (\textbf{ID}, corso references corso(ID), anno)\\
ESAME (\textbf{corso} references occorrenza\_corso(ID), \textbf{docente} references
docente(codice), \textbf{studente} references studente(matricola), voto, data)

\subsection{Testo esercizio}

\begin{enumerate}
\item
Calcolare la media ottenuta in Informatica Generale da chi non ha superato Matematica I.
\item
Identificare gli studenti che hanno ottenuto un voto superiore alla media calcolata al
punto 1.
\item
Calcolare il numero massimo di volte che uno studente non ha superato l'esame di
Informatica Generale
\item
Determinare gli studenti interessati dalla soluzione del punto 3.
\item
Calcolare quando uno studente ha superato un esame con un docente con cognome uguale allo
studente.
\item
Determinare per ogni docente e per ogni anno quanti esami il docente ha fatto
in
quel particolare anno e  la percentuale di esami con successo sugli esami
totali.
\item
Per quali corsi esiste un anno in cui ci sono stati meno esami superati
rispetto
all'anno precedente?
\item
Calcolare la media ottenuta in Informatica Generale da chi ha fallito almeno
una volta Matematica I.
\end{enumerate}


\subsection{Interrogazioni}



\begin{enumerate}
\item
Un primo approccio prevede di utilizzare il costrutto \texttt{NOT IN}, mentre
un secondo approccio {\`e} basato sull'operatore \texttt{EXCEPT}: si noti che
quest'ultimo {\`e} applicabile solo a select conformi, ovvero il cui risultato
abbia stesso schema. In entrambi i casi {\`e} fondamentale incrociare le tabelle
\texttt{esame}, \texttt{occorrenza\_corso}, \texttt{corso} per associare ad
ogni esame il nome dell'insegnamento a cui si riferisce.
\begin{verbatim}
create view esami_superati as
select matricola, nome, cognome, voto, corso
from studente, esame, occorrenza_corso, corso, persona
where persona.ID=studente.persona AND
  occorrenza_corso.corso=corso.ID AND
  esame.corso=occorrenza_corso.ID AND
  esame.studente=studente.matricola AND
  voto IS NOT NULL;
\end{verbatim}
\begin{verbatim}
select avg(voto)
from esami_superati
where corso='Informatica Generale' AND
matricola NOT IN (
  select matricola
  from esami_superati
  where corso='Matematica I'
);
\end{verbatim}
Nel secondo caso decido di utilizzare EXCEPT al fine di calcolare gli studenti
che hanno superato Informatica Generale ma non Matematica I.
\begin{verbatim}
create view infgen_no_mate(matricola) as (
select studente
from esame
where corso='Informatica Generale' AND voto is not null

except

select studente
from esami, occorrenza_corso, corso
where nome='Matematica I' and
      esami.corso=occorrenza_corso.id and
          occorrenza_corso.corso=corso.id and
          voto is not null
)

select avg(voto)
from esami, occorrenza_corso, corso, infgen_no_mate
where nome='Informatica Generale' and
      esami.corso=occorrenza_corso.id and
          occorrenza_corso.corso=corso.id and
          esami.studente=infgen_no_mate.matricola
\end{verbatim}
\item
\begin{verbatim}
select nome, cognome
from esami_superati
where corso='Informatica Generale' AND
voto > ALL (select avg(voto)
    from esami_superati
    where corso='Informatica Generale' AND
    matricola NOT IN (
      select matricola
      from esami_superati
      where corso='Matematica I'
    )
);
\end{verbatim}
oppure
\begin{verbatim}
create view esame_infgen_no_mate(matricola, voto) as (
        select esami.studente, esami.voto
    from esami, occorrenza_corso, corso, infgen_no_mate
    where nome='Informatica Generale' and
      esami.corso=occorrenza_corso.id and
          occorrenza_corso.corso=corso.id and
          esami.studente=infgen_no_mate.matricola and
          voto is not null
)

select matricola
from esame_infgen_no_mate
where esami.voto >= all (
                            select avg(voto)
                        from esame_infgen_no_mate)
\end{verbatim}
\item
\begin{verbatim}
create view infgen_nonsup as
select matricola, count(*) as numero
from esame, occorrenza_corso, corso
group by matricola
where occorrenza_corso.corso=corso.ID AND
  esame.corso=occorrenza_corso.ID AND
  corso.nome='Informatica Generale' AND
  voto IS NULL;
\end{verbatim}
\begin{verbatim}
select max(numero)
from infgen_nonsup;
\end{verbatim}
\item
\begin{verbatim}
select matricola, nome, cognome
from infgen_nonsup join persone on persone.matricola= infgen_nonsup.matricola
where numero >= ALL
    (select numero
    from infgen_nonsup
);
\end{verbatim}
\item
\begin{verbatim}
select *
from studente, esame, occorrenza_corso, corso, persona D, persona S, docente
where S.ID=studente.persona AND
  D.ID=docente.persona AND
  esame.docente=docente.codice AND
  esame.corso=occorrenza_corso.ID AND
  esame.studente=studente.matricola AND
  voto IS NOT NULL AND
  S.cognome=D.cognome;
\end{verbatim}
\item
\begin{verbatim}
create view esamiripartiti as
select codice, count(*) as totali, anno
from esame
group by codice. anno

create view esamiripartitipassati as
select codice, count(*) as totali, anno
from esame
where voto is not null
group by codice. anno

select *, passati/totali as percentuale
from esamiripartitipassati join esamiripartiti
where
esamiripartitipassati.anno = esamiripartiti.anno and
esamiripartitipassati.codice = esamiripartiti.codice
\end{verbatim}
\item
\begin{verbatim}
select distinct corso
from esamiripartitipassati A, esamiripartitipassati B,
where
A.anno < B.anno and
A.corso = B.corso and
A.totali > B.totali
\end{verbatim}
\end{enumerate}


\section{Biblioteca}

\subsection{Testo esercizio}

Si desidera modellare i dati relativi alla gestione di una biblioteca. Ogni libro viene
identificato univocamente dal codice ISBN, ed \`e caratterizzato dal fatto di avere uno o
pi\`u autori (non verranno effettuate ricerche sugli autori), titolo, codice dewey e genere.

Gli utenti hanno un codice identificativo e viene conservato nome, cognome e numero di
telefono (quest'ultimo dato \`e facoltativo). Per ogni libro preso in prestito
segnamo la data di inizio prestito, la data di scadenza e la data in cui \`e
stato effettivamente restituito (quest'ultimo dato viene posto a NULL se il libro non \`e
ancora stato restituito).

Modellare la situazione descritta con modello ER. Descrivere la query SQL, e le tabelle
coinvolte in tale query, che restituisce l'elenco delle persone che, per almeno 3 volte,
non hanno restituito un libro entro la scadenza. Tale esercizio deve essere
risolto secondo due versioni: la prima considera soltanto il fatto che ci siano
state tre restituzioni in ritardo, la seconda anche il fatto che alla data
odierna ci potrebbero essere dei libri in prestito anche se la data di scadenza
\`e gi\`a passata.


Come si modifica lo schema ER
se non si ammette la possibilit\`a che un utente prenda in prestito lo stesso
libro pi\`u di una volta?

Se avessimo il vincolo che un utente pu\`o avere al massimo 5 libri in
prestito contemporaneamente, come avremmo codificare tale vincolo?

Si calcoli quale libro \`e stato restituito in ritardo pi\`u volte.

\subsection{ER}
\begin{center}
\pgfimage[width=15cm]{DB-Biblio}
\end{center}

Si \`e deciso di utilizzare un'entit\`a per Prestito, invece di una relazione, perch\`e si
ritiene rilevante rappresentare i prestiti di un libro dismesso dalla libreria, ci\`o in
quanto viene richiesto di conteggiare le persone che hanno consegnato un libro in ritardo
per almeno 3 volte. Questo fatto presuppone che Prestito abbia valore a prescindere
dall'esistenza di Libro, e quindi un'entit\`a \`e preferibile. Si noti che il
vincolo che il numero di telefono sia opzionale \`e rappresentabile solo con il
codice SQL di creazione della tabella.

\subsection{Tabelle}

\noindent
LIBRO (\textbf{ISBN}, Autore, Titolo, Dewey, Genere)\\
UTENTE (\textbf{codice}, nome, cognome, numerotel)\\
PRESTITO (\textbf{utente} references utente(codice), \textbf{libro} references libro(ISBN),
\textbf{scadenza}, restituito, inizio)

\subsection{Query}

\begin{verbatim}
select codice, nome, cognome
from utente join prestito on utente(codice)=prestito(utente)
where codice in
    (select codice
    from prestito
    where restituito>scadenza
    group by codice
    having count(*)>=3);
\end{verbatim}

\noindent
La versione completa segue:

\begin{verbatim}
select utente, nome, cognome
from utente join prestito on utente(codice)=prestito(utente)
where utente in
    (select utente
    from prestito
    where (restituito>scadenza or today()>scadenza and restituito is null)
    group by utente, nome, cognome
    having count(*)>=3);
\end{verbatim}


\newpage
\section{Rete ferroviaria}
\subsection{Testo}
Si vuole rappresentare l'insieme di dati riguardanti una rete ferroviaria. Ogni citt\`a
viene identificata univocamente tramite il proprio nome, ed ogni tratta consiste in una
citt\`a di partenza, una citt\`a di arrivo ed una lunghezza epressa in km. Si suppone che non
esistano stazioni intermedie fra le due citt\`a che compongono una tratta.

Ogni biglietto emesso ha un numero identificativo progressivo ed ha un prezzo, dipendente
dalla massima percorrenza possibile (tale percorrenza non viene memorizzata). Per ogni
volta che una tratta viene percorsa si tiene traccia dell'orario di partenza e dell'orario
di arrivo. Al momento
della convalida del biglietto viene immesso l'insieme delle tratte che compongono il
viaggio per cui il biglietto viene utilizzato e data/ora di partenza e arrivo del viaggio.

Si scriva la query SQL che determina su quale tratta sono stati utilizzati pi\`u biglietti
nel mese di Ottobre 2001.

\subsection{ER}
\begin{center}
\pgfimage[width=15cm]{es2}
\end{center}

Ogni tratta viene univocamente determinata dalle citt\`a di arrivo e partenza. Inoltre
l'introduzione dell'entit\`a \texttt{occorrenza\_tratta} permette di non duplicare
l'informazione relativa all'orario di partenza e arrivo da associare ad ogni biglietto:
infatti la relazione \texttt{biglietto\_utilizzato} permette di individuare tale
informazione.

\subsection{Tabelle}

\noindent
CITTA'(\textbf{nome})\\
TRATTA(\textbf{inizio} references citta'.nome, \textbf{fine} references citta'.nome,
lunghezza)\\
BIGLIETTO(prezzo, \textbf{ID})\\
OCCORRENZA\_TRATTA(\textbf{inizio} references citta'.nome, \textbf{fine} references
citta'.nome, \textbf{partenza}, arrivo)\\
BIGLIETTO\_UTILIZZATO(\textbf{ID} references biglietto, \textbf{inizio} references
citta'(nome), \textbf{fine} references citta'(nome), \textbf{partenza} references occorrenza\_tratta)

\subsection{Query}
\begin{verbatim}
create view usati_ott as
select count(*) as quanti, inizio, fine
from biglietto_utilizzato
where partenza <= '31/10/2000' and partenza >= '01/10/2000'
group by partenza, arrivo;

select inizio, fine
from usati_ott
where quanti >= ALL
    (select quanti
     from usati_ott);
\end{verbatim}

\newpage
\section{Campionato di calcio}

\subsection{Testo}
Un campionato di calcio viene disputato tra un insieme di squadre. Ogni incontro \`e svolto
fra due squadre, una detta ospitante ed una detta ospite. Per ogni incontro, oltre alle
due squadre coinvolte, vengono registrati il risultato (inteso come numero di reti segnate
dalle ognuna delle due squadre), la data ed i giocatori che hanno
realizzato ogni rete, si memorizza anche il minuto della partita in cui \`e stata
realizzata la rete.

In un campionato ogni squadra incontra ogni altra squadra due volte, di cui una come
squadra ospite ed una come squadra ospitante.

Di ogni calciatore viene registrato nome, cognome e data di nascita. Si assume inoltre che
un giocatore non possa cambiare squadra durante il campionato.

Si progetti una base di dati per rappresentare la situazione sopra descritta, sia come
modello ER che come insieme di tabelle.

Si scriva una query SQL che restituisce quale squadra ha vinto pi\`u partite, una
query per determinare il calciatore che ha segnato pi\`u reti ed una query per
determinare in  quale partita si \`e realizzato il numero maggiore di reti (somma
fra le reti della squadra ospite e quelli della squadra ospitante).


Come viene modificato lo schema se una squadra pu\`o incontrare un'altra squadra
pi\`u di due volte?

Come viene modificato lo schema se un giocatore pu\`o cambiare squadra durante il
campionato?

\subsection{ER}
\begin{center}
\pgfimage[width=15cm]{es3}
\end{center}

\subsection{Tabelle}

\noindent
SQUADRA (\underline{nome})\\
PARTITA (\underline{ospite} references squadra(nome), \underline{ospitante} references
squadra(nome), data)\\
GIOCATORE (\underline{nome}, \underline{cognome}, \underline{natoil}, giocain references
squadra(nome))\\
RETE (\underline{nome} references giocatore, \underline{cognome} references giocatore,
\underline{data\_di\_nascita} references giocatore, \underline{ospite}
references squadra(nome), \underline{ospitante} references squadra(nome), minuto)

\subsection{Query}

\begin{verbatim}
create view retipartitasquadra as
  select ospite, ospitante, giocain as squadra, count(*) as quanti
  from partita, rete, giocatore
  where giocatore.nome =rete.nome and
        giocatore.cognome =rete.cognome and
        giocatore.natoil =rete.natoil and
        partita.ospite = rete.ospite and
        partita.ospitante = rete.ospitante
  group by partita.ospite, partita.ospitante, giocatore.gioca_in


create view risultati as
  select R1.quanti as retiospitante, R2.quanti as retiospite
  from retipartitasquadra R1, retipartitasquadra R2
  where R1.ospite = R2.ospite and
        R1.ospitante = R2.ospitante and
        R1.squadra <> R2.squadra
        R1.ospitante = R1.squadra



create view vittorie as
   select ospite as squadra
   from partita
   where reti_ospite>reti_ospitante
 union all
   select ospitante as squadra
   from partita
   where reti_ospite<reti_ospitante

select nome
from vittorie
group by squadra
having count(*) > ALL (
    select count(*)
    from vittorie
    group by squadra);
\end{verbatim}


\subsection{Varie}
\texttt{Rete} deve essere una entit{\`a}, in quanto {\`e} ragionevolmente atteso che un
giocatore segni pi{\`u} di una rete in una partita, pertanto l'attributo \texttt{minuto}
deve essere parte della chiave.

Se il numero di partite fra una coppia di squadre non \`e limitato, allora
\texttt{data} deve fare parte della chiave dell'entit\`a \texttt{partita}.

Se un giocatore pu\`o cambiare squadra, allora la cardinalit\`a della relazione
\texttt{gioca in} non pu\`o pi\`u essere $(1,1)$. In questo caso deve anche
diventare una tabella a s\`e stante. Inoltre non {\`e} pi{\`u} possibile inferire il
risultato dalla relazione rete, in quanto non si pu{\`o} stabilire a favore di quale
squadra sia stata realizzata una rete. Diventa quindi necessario (1)
introdurre una relazione \texttt{partecipa} che lega le entit{\`a} \texttt{giocatore} e
\texttt{partita}, oppure (2) introdurre una relazione \texttt{a favore} che lega le entit{\`a} \texttt{rete} e
\texttt{squadra}. Dalle numerosit{\`a} ragionevolmente attese, quest'ultima
ipotesi mi sembra preferibile.

\newpage
\section{Servizio di fornitura di energia elettrica}

\subsection{Testo}

Viene fornita energia elettrica ad un insieme di utenze, con la possibilit\`a che nuove
utenze vengano aggiunte nel tempo. Per ogni utenza registriamo, oltre ad un codice
identificativo, nome, cognome e
l'attivazione dell'utenza. Quando un contratto viene attivato, si registra un codice
identificativo, la data in cui
sono stati completati i lavori ed il tecnico responsabile dell'impianto.

Per ogni utenza vengono inoltre memorizzati i dati storici relativi ad ogni lettura del
contatore. Dopo ogni lettura viene emessa una fattura di cui memorizziamo la lettura,
prima e dopo il periodo di riferimento della fattura stessa, la data di emissione e
l'intestatario. Nel caso in cui non sia stata possibile la lettura, la fattura viene
emessa senza avere una lettura di riferimento.

Disegnare lo schema ER e le tabelle.

Query 1: l'ultimo importo pagato da ogni utente di cognome Rossi.

Query 2: le letture da mettere in fattura del 01.07.2001 del cliente con codice 003 (tale
fattura ha una lettura di riferimento).

Modificare lo schema per incorporare la seguente modifica: l'attivazione avviene
tramite una sequenza di sopralluoghi effettuati presso la stessa utenza. Ogni
sopralluogo viene fatto da un tecnico potenzialmente diverso. L'attivazione
avviene quando l'ultimo sopralluogo verifica che sono verificate le condizioni
per l'attivazione.
\subsection{ER}
\begin{center}
\pgfimage[width=15cm]{es4}
\end{center}

\subsection{Tabelle}

\noindent
UTENZA (nome, cognome, \underline{ID}, attivazione unique, data\_attivazione,
tecnico references tecnico(ID))\\
TECNICO (nome, cognome, \underline{ID})\\
FATTURA (\underline{numero}, data, importo, quantita', utente references utenza(ID) null,
data\_lettura null)\\

\subsection{Query}

\begin{verbatim}
create view fattrossi as
    select ID, importo, data_fattura
    from utenza join fattura on utenza.ID=fattura.utente
    where cognome='Rossi';

select importo, data
from fattrossi F
where data >= ALL
    (select data
     from fattrossi F1
     where f.numero=f1.numero);
\end{verbatim}

\hrule

\begin{verbatim}
select lettura
from fattura
where data<'01/01/2001' AND
  data >= ALL (
    select data
    from fattura
    where data<'01/01/2001');
\end{verbatim}

\section{Conferenza scientifica}

\subsection{Testo}

Una conferenza viene organizzata nel seguente modo: uno (o pi\`u) autori sottopongono un
articolo affinch\`e venga valutato e, alla fine del processo di valutazione,
accettato oppure rifiutato alla conferenza. Ad ogni
articolo viene assegnato un numero progressivo, e si tiene traccia del titolo e degli
autori. Per ogni autore si memorizzano il nome, il cognome e l'indirizzo di posta
elettronica che identifica univocamente ogni autore.

La conferenza ha un comitato direttivo i cui membri possono (ma non sono obbligati)
sottoporre articoli. Ogni articolo viene valutato da almeno 3 ed al pi\`u 5 revisori, di cui
esattamente uno deve essere membro del comitato direttivo. Si noti che non tutti
i membri del comitato direttivo devono essere revisori.

Nessuno pu\`o essere revisone di un articolo di cui \`e autore. Al termine della revisione,
ogni revisore decide un voto da assegnare agli articoli valutati. Solo alcuni articoli
vengono accettati alla conferenza, conseguentemente lo stato di ogni articolo \`e
accettato, oppure rifiutato, oppure ancora in revisione. Di ogni revisore o
membro del comitato direttivo
vengono memorizzati numero di telefono e fax, oltre ai dati identificativi di un autore.

Descrivere modello ER, tabelle e codice SQL per le sequenti interrogazioni.
\begin{enumerate}
\item
Quale autore ha avuto il massimo numero di articoli accettati?
\item
Quali revisori non hanno ancora inviato la loro valutazione?
\item
Visualizzare la media delle valutazioni di ogni articolo.
\item
Visualizzare la media delle valutazioni degli articoli scritti da ogni singola persona.
\item
Visualizzare i 20 articoli con la media pi\`u alta.
\end{enumerate}
Come si modifica lo schema se i membri del comitato direttivo non possono essere
autori di articoli sottoposti alla conferenza?
\subsection{ER}
\begin{center}
\pgfimage[width=15cm]{es5}
\end{center}

\subsection{Tabelle}

\noindent
ARTICOLO (\underline{ID}, titolo, stato)\\
PERSONA (nome, cognome, \underline{email}, telefono, fax, is\_CD)\\
SOTTOPONE (lavoro REFERENCES artiolo(ID), autore REFERENCES autore(email))\\
VALUTA (revisore REFERENCES persona(email), lavoro REFERENCES artiolo(ID))

\subsection{SQL}

1.
\begin{verbatim}
create view lavori_accettati as
  select id, titolo, nome, cognome
  from articolo, sottopone, autore
  where articolo.ID=sottopone.lavoro AND
     sottopone.autore=autore.email AND
    accettato;

select email, nome, cognome
from lavori_accettati
group by email, nome, cognome
having count(*) >= ALL
  (select count(*)
  from lavori_accettati
  group by email);
\end{verbatim}

2.
\begin{verbatim}
select distinct nome, cognome, email
from persona join valuta on persona.email=valuta.revisore
where voto is null;
\end{verbatim}

3.
\begin{verbatim}
select avg(voto), titolo, ID
from articolo join valuta on articolo.ID=valuta.articolo
group by ID;
\end{verbatim}

4.
\begin{verbatim}
select avg(voto), nome, cognome, email
from valuta, sottopone, articolo
where articolo.ID=sottopone.lavoro AND
    valuta.lavoro=articolo.ID
group by email;
\end{verbatim}


\newpage
\section{Catena di cinema}

\subsection{Testo}

Una catena di cinema ha sale in varie citt\`a. Di ogni cinema si memorizza la
citt\`a e la via in cui si trova. Alcuni cinema sono multisala, mentre altri
contengono un'unica sala. Di ogni sala si memorizza la capienza totale.

Ogni film ha un titolo (si assuma che non esistano due film con stesso titolo),
l'anno di produzione
un regista e vari attori. Per ogni persona esiste un codice identificativo.

Viene inoltre tenuta traccia di ogni proiezione effettuata, ovvero del film
proiettato, in quale sala \`e avvenuta la registrazione, ed il numero di biglietti
venduti.

Disegnare lo schema ER e le tabelle.

Query 1: l'elenco dei film in cui il numero dei biglietti totali venduti \`e stato almeno
l'80\% della capienza totale (sommata su tutte le proiezioni del film).

Query 2: l'elenco dei film che hanno fatto il tutto esaurito in almeno due
spettacoli tenuti nella stessa citt\`a.

Query 3: l'elenco degli registi che si sono diretti in un film.

Query 4: I film che hanno venduto pi\`u biglietti.

Query 5: l'elenco dei film che hanno fatto il tutto esaurito in almeno due citt\`a.
\subsection{ER}
\begin{center}
\pgfimage[width=15cm]{es-cinema}

\end{center}

\subsection{Tabelle}
\noindent
PERSONE (\underline{ID}, cognome, nome, data\_nascita)\\
FILM (\underline{titolo}, regista references persone(ID))\\
CINEMA (\underline{citt\`a}, \underline{via})\\
SALA (ID, citta references CINEMA(citta), via references
CINEMA(via), capienza)\\
PROIEZIONE (\underline{film}  references FILM(titolo), \underline{citta\_sala}
references CINEMA(citta), \underline{sala} references SALA(ID), inizio, data)\\
ATTORE (\underline{persona}   references PERSONA(ID), \underline{film}   references
FILM(titolo))

\subsection{SQL}

Query 1.
\begin{alltt}
create view venditeproiezione as
  select titolo, sum(capienza) as capienzatot, sum(venduti) as vendutitot
  from proiezione, sala
  where sala.ID=proiezione.sala
  group by titolo;

select titolo
from venditeproiezione
where vendutitot/capienzatot>=0.8;
\end{alltt}

Query 2.
\begin{alltt}
create view esauriti as
  select titolo, citt\`a, via, count(*) as numero
  from proiezione, sala
  where sala.citt\`a=proiezione.citt\`a AND
      sala.ID=proiezione.sala AND proiezione.venduti=sala.capienza;
  group by titolo, citt\`a

select titolo
from esauriti
where numero>=2;
\end{alltt}

Query 3.
\begin{alltt}
select nome, cognome
from persone, attore, film
where film.regista=attore.persona AND
    attore.film=film.titolo AND attore.persona=persone.id;
\end{alltt}

Query 4.
\begin{alltt}

select titolo
from venditeproiezione
where vendutitot >= ALL (
select vendutitot
from venditeproiezione);
\end{alltt}


Query 5.
\begin{alltt}

select E1.titolo
from esauriti E1, esauriti E2
where E1.titolo=E2.titolo and
      E1.citt\`a <> E2.citt\`a
\end{alltt}


\end{document}
%%% Local Variables:
%%% mode: latex
%%% TeX-PDF-mode: t
%%% End:
